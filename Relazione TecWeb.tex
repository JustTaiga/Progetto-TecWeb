% !TeX spellcheck = it_IT
%Alberto Bobbo, Michele Bortone, Enrico Marcato
\documentclass[10pt, a4paper]{article}

\usepackage[scaled]{helvet}

\usepackage[utf8]{inputenc}
\usepackage[T1]{fontenc}
\usepackage[italian]{babel}

\usepackage{graphicx}
\usepackage{fix-cm}
\newcommand{\bigsize}{\fontsize{35pt}{20pt}\selectfont}
\newcommand{\mediumsize}{\fontsize{30pt}{20pt}\selectfont}
\newcommand{\normsize}{\fontsize{15pt}{10pt}\selectfont}


%crea uno spazio, utile perchè gli spazi extra dopo le macro
%vengono rimossi. Questo comando permette di inserirne uno.
\def\space{ }

\usepackage{float}
\usepackage{caption}

\usepackage{amsmath}
\usepackage{mathtools}

\usepackage{multirow}
%crea una cella per le tabelle in grado di andare a capo con \newline
%https://tex.stackexchange.com/questions/12703/how-to-create-fixed-width-table-columns-with-text-raggedright-centered-raggedlef
\usepackage{array}
\newcolumntype{L}[1]{>{\raggedright\let\newline\\\arraybackslash\hspace{0pt}}m{#1}}
\newcolumntype{C}[1]{>{\centering\let\newline\\\arraybackslash\hspace{0pt}}m{#1}}
\newcolumntype{R}[1]{>{\raggedleft\let\newline\\\arraybackslash\hspace{0pt}}m{#1}}

%puntini per l'indice
\usepackage{tocloft}
\renewcommand\cftsecleader{\cftdotfill{\cftdotsep}}


%https://tex.stackexchange.com/questions/4503/how-do-i-specify-color-in-rgb-using-hypersetup-in-hyperref
\usepackage{url}
\usepackage{breakurl}
\usepackage[colorlinks=true]{hyperref}
\usepackage[hyperref]{xcolor}
\definecolor{UniPD}{RGB}{155, 0, 20}
\definecolor{Crema}{RGB}{220, 197, 149}
\definecolor{LinkNormNoClick}{RGB}{0, 0, 238}
\definecolor{LinkNormClick}{RGB}{69, 123, 157}
\hypersetup{colorlinks,breaklinks,
	urlcolor=UniPD,
	linkcolor=UniPD}

%per alcune liste
\usepackage{blindtext}
\usepackage{scrextend}
\addtokomafont{labelinglabel}
{\sffamily}


\newcommand{\Componenti}{Alberto Bobbo \newline Michele Bortone \newline
	Enrico Marcato}
\newcommand{\Referente}{Alberto Bobbo \newline alberto.bobbo@studenti.math.unipd.it}
\newcommand{\Gruppo}{Bobbo, Bortone, Marcato}
\newcommand{\Titolo}{Relazione Progetto Tecnologie Web}

\usepackage{lastpage} %info sul # dell'ultima pagina del documento
\usepackage{fancyhdr} %per modificare dimensioni,margini, intestazioni e righe a piè di pagina
\fancypagestyle{plain}{
	% cancella tutti i campi di intestazione e piè di pagina
	\fancyhf{}
	
	\lfoot{ %piè di pagina
		\Titolo{} \ - \textit{\Gruppo{}}
	}
	\rfoot{Pagina \thepage{} di \pageref{LastPage}} %es: pag: 4 di 10
	
	%linea orizzontale alle posizioni top e bottom della pagina
	\renewcommand{\headrulewidth}{0pt}  
	\renewcommand{\footrulewidth}{0.3pt}
}
\pagestyle{plain}

\usepackage{listings}

\definecolor{background}{RGB}{46, 46, 46}  %%nero
\definecolor{string}{RGB}{230, 219, 116} %%giallo
\definecolor{comment}{RGB}{117, 113, 94} %%grey
\definecolor{normal}{RGB}{248, 248, 242} %%bianco
\definecolor{purple}{RGB}{242, 16, 114} %%purple
\definecolor{identifier}{RGB}{166, 226, 46} %%verde
\definecolor{blue}{RGB}{0, 102, 255}


\lstdefinelanguage{HTML5}{
	sensitive=true,
	keywords={
		%% JavaScript
		typeof, new, true, false, catch, function, return, null, catch, switch, var, if, in, while, do, else, case, break, foreach, as,
		%% HTML
		html, meta, style, head, body, script, canvas, h1, h2, h3, h4, h5, h6, table, thead, tbody, tfoot, p, a, div, input, form, tr, th, label, ?php, ?, 
		%% CSS
		border:, transform:, -moz-transform:, transition-duration:, transition-property:,
		transition-timing-function:
	},
	% http://texblog.org/tag/otherkeywords/
	keywords=[2]{<, >, \, /, />, </ },  %%tag
	keywords=[3]{href, title, label, aria-label, lang, for, tabindex, placeholder, id, type, value, class, scope, }, %%attributi
	ndkeywords={class, export, boolean, throw, implements, import, this},
	keywords=[4]{echo, print\_ordinable\_th, REQUIRED},
	comment=[l]{//},
	% morecomment=[s][keywordstyle]{<}{>},  
	morecomment=[s]{/*}{*/},
	morecomment=[s]{<!}{>},
	morestring=[b]',
	morestring=[b]",    
	alsoletter={-},
	alsodigit={:}
}

\lstset{
	backgroundcolor=\color{background},
	tabsize=4,    
	language=HTML5,
	basicstyle=\ttfamily\linespread{1.15}\footnotesize,
	upquote=true,
	aboveskip={1.5},
	columns=fixed,
	showstringspaces=false,
	extendedchars=true,
	inputencoding=utf8,
	breaklines=true,
	prebreak = \raisebox{0ex}[0ex][0ex]{\ensuremath{\hookleftarrow}},
	frame=none,
	numbers=left,
	numbersep=5pt,	
	showtabs=false,
	showspaces=false,
	showstringspaces=false,
	basicstyle=\tiny\color{normal},
	identifierstyle=\color{normal},
	keywordstyle=\color{purple},
	keywordstyle=[2]\color{normal},
	keywordstyle=[3]\color{identifier},
	keywordstyle=[4]\color{blue},
	commentstyle=\color{comment},
	stringstyle=\color{string},
	numberstyle=\tiny\color{background}.
}



\begin{document}


\begin{titlepage}
\centering

\includegraphics[width=50mm]{Images/logo.png}
\vspace*{30px}
{\Large \\ \textbf{RELAZIONE PROGETTO TECNOLOGIE WEB}\\}
\vspace*{30px}

\bgroup
\def\arraystretch{1.3}
\centering
\begin{tabular}{c|L{5cm}}
\multicolumn{2}{c}{\textbf{Gruppo} } \\ \hline
  Componenti & \Componenti{} \\
  Referente & \Referente{}
\end{tabular}
\egroup

\vspace*{80px}


\hypersetup{hidelinks}
\bgroup
\def\arraystretch{1.3}
\centering
\begin{tabular}{c}
\multicolumn{1}{c}{\textbf{Indirizzo Web Del Sito} } \\
  \url{tecweb.studenti.math.unipd.it/abobbo}
\end{tabular}
\egroup

\vspace*{80px}

\begin{tabular}{c|L{4cm}}
\multicolumn{2}{c}{\textbf{Credenziali Admin} } \\ \hline
  Username & admin@progetto.com \\
  Password & admin
\end{tabular}
\quad
\begin{tabular}{c|L{4cm}}
\multicolumn{2}{c}{\textbf{Credenziali Utente} } \\ \hline
  Username & gino@progetto.com \\
  Password & ciaog
\end{tabular}

\vspace*{10px}

\end{titlepage}


\newpage
\hypersetup{hidelinks}
\tableofcontents
\newpage







\section{Presentazione sito}
Il progetto in questione consiste nella realizzazione di un sito web che ha lo scopo di presentare l’evento musicale “Home Festival”. Il sito deve essere in grado di fornire all’utente tutte le informazioni di cui ha bisogno per poter assistere ai concerti che si terranno durante l’evento. Potrà informarsi sul luogo, gli orari e gli artisti presenti.
Inoltre, dopo una breve registrazione, gli utenti potranno commentare scambiandosi impressioni e pareri sulle ultime news e gli ultimi articoli pubblicati dagli amministratori.

\section{Utenti destinatari}
Il sito è destinato a tutti gli utenti che intendono partecipare all’Home Festival o che comunque desiderano avere informazioni riguardo all’evento.  
All’interno del footer sono inoltre presenti alcuni contatti, nel caso l’utente volesse ricevere informazioni aggiuntive, e i link alle pagine dei principali social network, dove vengono inseriti gli ultimi aggiornamenti riguardanti l’evento.

\section{Tipi di utenti}
\section{Progettazione}
\subsection{Organizzazione dell'informazione}
\subsection{Aree del sito}
\subsection{Struttura organizzativa}
\section{Accessibilità}
\subsection{Perceivable}
\subsubsection{Text Alternatives}
\subsubsection{Time-based Media}
\subsubsection{Adaptability}
\subsection{Understandable}
\subsubsection{Legibility}
\subsubsection{Predictability}
\subsection{Robust}
\subsubsection{Compatibility}
\subsection{Operable}
\subsubsection{Keyboard Accessibility}
\subsubsection{Time Availability}
\subsubsection{Epilepsy}
\subsubsection{Navigability}
\subsection{Visione del sito da parte di individui con disturbi visivi}
\subsubsection{Deuteranopia}
\subsubsection{Protanopia}
\subsubsection{Tritanopia}
\section{Usabilità}
\subsection{Link}
\subsection{Navbar}
\section{Amministratore, utente}
\subsection{Amministratore}
\subsection{Utente}
\section{Gerarchia dei file}
\section{HTML5}
\section{PHP}
\section{Javascript}
\section{Validazione}
\section{Compatibilità browser}
\subsection{Compatibilità con Internet Explorer}
\subsubsection{Internet Explorer 8}
\subsubsection{Internet Explorer 11}
\subsection{Compatibilità con Microsoft Edge}
\subsection{Compatibilità con Opera}
\subsection{Compatibilità con Safari}
\subsection{Compatibilità con Chrome}
\subsection{Compatibilità con Firefox}
\subsection{Dispositivi Mobili}
asdasdasd
\section{Organizzazione}
\begin{labeling}{alligator}
	\item[\textbf{Alberto Bobbo}] \item[] %serve perché altrimenti il primo elemento della lista inizia su questa linea
		\begin{itemize}
			\item{Creazione e reperimento dei contenuti multimediali}
			\item{Codice HTML }
			\item{Codice CSS per la presentazione}
			\item{Validazione codice e correzione errori}
		\end{itemize}
	\item[\textbf{Michele Bortone}] \item[]
		\begin{itemize}
			\item{Codice HTML}
			\item{Creazione e gestione del database}
			\item{Codice PHP per la gestione dinamica dei contenuti}
			\item{Test funzionalità del sito su vari dispositivi e browser}
		\end{itemize}
	\item[\textbf{Enrico Marcato}] \item[]
		\begin{itemize}
			\item{Codice HTML }
			\item{Codice CSS per la presentazione}
			\item{Codice CSS per la stampa}
			\item{Codice PHP per la sessione utente   }
		\end{itemize}
\end{labeling}







































\end{document}